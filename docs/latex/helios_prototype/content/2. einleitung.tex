\section{Einleitung}

Wir programmieren einen \textit{Geometry Wars}\footnote{Siehe~\cite[]{WikipediaGeometryWars}} Klon in C++.
Bei der Umsetzung orientieren wir uns teilweise an einer in der Industrie weit verbreiteten, von Gregory in~\cite[]{Gre19} beschriebenen \textit{Game Engine Architektur}, reduzieren jedoch bewusst die Anzahl der Abstraktionsschichten und verzichten auf Systeme wie Tooling, Sound oder Scripting (vgl.~\cite[Figure~1.16, 39]{Gre19}).
Der technische Unterbau entspricht damit eher einem Framework als einer Game Engine.
Das eigentliche Spiel soll die Schnittstellen des Frameworks für benötigte Hardware und andere Subsysteme nutzen, und vom Framework als \textit{Black Box}~\cite[]{RB88} betrachtet werden.

\begin{figure}[tbp]
    \centering
    \includegraphics[width=0.5\columnwidth]{img/helios_logo}
    \caption{Das helios Projekt-Logo.}
    \label{fig:helios_logo}
\end{figure}

In den nachfolgenden Abschnitten wird die Architektur von dem mit \textbf{helios}\footnote{
    Helios, der ``scharf vor allen mit strahlenden Augen umherblickt`` (\textit{Homers Ilias}, 14. Gesang), ist in der griechischen Mythologie der Sonnengott.
} bezeichneten Framework vorgestellt.
Dabei berücksichtigen wir seinen prototypischen Entwicklungsstand:
Die Software wird im Rahmen eines agilen \textit{Tracer Bullet Development}-Prozess~\cite[50 f.]{TH20} entwickelt.
Dadurch soll das Gesamtsystem bei Bedarf kurzfristig angepasst werden können.
Hieraus leiten sich Anforderungen an die Architektur ab.
Projektdaten und -ziele stellen wir in Abschnitt~\ref{sec:projektdaten} gesondert vor.\par

Wir betrachten außerdem Probleme und Schwierigkeiten, die bisher bei der Umsetzung aufgetreten sind.
Diese ergeben sich unter anderem durch den Umstand, dass das in vergleichsweise kurzer Zeit zu erstellende Softwareprodukt nicht nur anhand objektiver Maße bewertet werden soll (etwa der Fehleranzahl, Software-Designentscheidungen, Kopplungsgrad der Objekte etc.), sondern auch anhand subjektiver Maße~\cite[385]{Bal08}.\\
Zwar ist keine Klassifizierung des fertigen Produkts nach Nutzungsqualitätsmerkmalen wie etwa ISO/IEC 9126 angedacht\footnote{``Nutzungsqualität nach ISO/IEC 9126``,~\cite[466]{Bal08}; Studien und Untersuchungen u.a. bei~\cite[]{AZMK17},~\cite[]{Ber10}}.
Dennoch soll das Ergebnis nicht nur technisch ausgereift, sondern für Anwenderinnen und Anwender angenehm zu bedienen sein.
Das Nutzererlebnis fassen wir im Weiteren formal eher unscharf, aber mit dem in der Spieleentwicklung etablierten Begriff des \textit{Game Feel}~\cite[]{Swi08} zusammen.

