\section{Projektziele und -daten}\label{sec:projektdaten}

Das Ziel der Projektarbeit ist es, einen stabil laufenden Prototyp des Twin-Stick Shooters\footnote{
Zur Genreeinteilung siehe auch \cite[]{GameDeveloper}
} \textit{Geometry Wars} (siehe Abbildung~\ref{fig:geometry_wars}) unter Verwendung der Programmiersprache \textit{C++} (23) und der OpenGL-API\footnote{siehe~\cite[]{OpenGLHomepage}
}  zu erstellen.\\

\noindent
Als grundsätzlich fertigzustellende Kernfeatures wurden vereinbart (siehe Anhang~\ref{sec:projektvorschlag}):

\vspace{2mm}
\begin{itemize}
    \itemsep0.5em
    \item Spielbares Level (2D-Grid)
    \item Time-Attack Modus mit einer Dauer von 3 Minuten
    \item Highscore-System
    \item Controller-Steuerung
    \item Stabile Framerate von mindestens 60 FPS bei gleichzeitiger Darstellung von über hundert Gegnern und Objekten
    \item Drei Gegnertypen mit einfacher KI (``Spawn and Chase``)
    \item Aufwertung des Game Feels durch grafische Effekte
    \item Compilierbar und lauffähig unter Windows 11
\end{itemize}
\vspace{2mm}

Die aufgeführten Punkte lassen nur vage Anforderungen erkennen, aus denen sich keine konkreten Abnahmekriterien ableiten lassen.
Ein \textit{Requirements Engineering} ist folglich auch nicht Bestandteil der Projektarbeit: Als ein Maß dient neben den o.a. funktionalen Kriterien vor allem auch das Original-Spiel, an dem sich die Umsetzung orientiert.\\

Insbesondere betrachten wir den \textit{Lernprozess} als ein wesentliches Ziel dieser Arbeit.
Für dessen Bewertung werden keine formalen Metriken herangezogen.
Stattdessen sollen die gewonnenen Erfahrungen und Ergebnisse im Anschluss schriftlich festgehalten und das hier vorliegende Dokument kritisch reflektiert werden.\\

Vor diesem Hintergrund wird das Vorgehen als eine Form des dynamischen Requirements Engineering ``mit deutlich größeren Freiheitsgeraden``~\cite[60]{MRP21} betrachtet: Das erwähnte \textit{Tracer Bullet Development}, das vor allem eine Plattform zur Integration bereitstellt, entspricht dem von \textit{Pflug et al.} mit ``Living Lab`` bezeichneten Prototyp, der im Weiteren nach agilen Methoden entwickelt wird (\textit{ebd., S. 61})\footnote{
    \textit{Kasurinen et al.} gehen in~\cite[]{KMS14} der Frage nach, ob das im Ingenieursbereich angesiedelte Anforderungsmanagement bei der stark durch kreative Prozesse beeinflussten Spieleentwicklung einen Platz hat. Sie zeigen, dass befragte Entwickler einen Teil der Anforderungen an ihr Spiel ganz bewusst erst durch \textit{User Testing} ableiten. Entsprechend stellt \textit{Games User Research} einen wichtigen Zweig der Spieleindustrie dar, der Unternehmen dabei unterstützt, Spiele-Erfahrung zu verstehen und zu verbessern (vgl.~\cite[26]{Zam18})
}, und ein Einbringen eigener kreativer Ideen erlauben soll.\\

\subsection{Zeitplan}

Wir stellen den Zeitplan vor und die aus dem Projektvorschlag übernommenen Meilensteinen, deren zeitliche Gliederung der Umsetzung dient\footnote{Aus der sequentiellen Auflistung soll kein Rückschluss auf ein Wasserfall-Modell gezogen werden}:

\vspace{2mm}
\begin{itemize}
    \itemsep0.5em
    \item \textbf{milestone\_1}: 20.10.2025: Bereitstellung der Applikationsschicht inkl. Implementierung des Event
    Systems, Input Managers sowie Anbindung an das Low Level API-Sub-Systems
    \item \textbf{milestone\_2}: 17.11.2025: Bereitstellung der Rendering Engine. Erster Entwurf des Spielfeldes
    samt Darstellung des Spielerschiffs.
    \item \textbf{milestone\_3}: 22.12.2025: Umsetzung der Physik und Spielereingabe zur Steuerung des Schiffs,
    Feuermechaniken
    \item \textbf{milestone\_4}: 19.01.2026: Implementierung der wichtigsten Spielregeln- und Mechaniken. Spiel
    fähiger Prototyp.
    \item \textbf{milestone\_5}: 09.02.2026: Bereitstellung des umgesetzten Prototyps. Feinschliff.
    \item \textbf{milestone\_6}: 16.03.2026: Abgabe der Dokumentation und Vorstellung des Projekts.
\end{itemize}
\vspace{2mm}

\noindent
Mit der Vorgabe des umzusetzenden Spielkonzeptes entfällt für uns dessen Prototyping.
Wir folgen \textit{Rollings und Morris} bei der Einordnung der gegenwärtigen Entwicklungsphase, die von den Autoren als \textit{hard-architecture design}-Phase bezeichnet wird (vgl.~\cite[628]{RM04}).\\

Das Projekt ist \textit{MIT} lizenziert auf Github veröffentlicht, Source Code und Ticketsystem sind öffentlich einsehbar~\cite[]{heliosgithub}.\\

Zum Zeitpunkt der Veröffentlichung dieses Dokumentes ist der erste Meilenstein abgeschlossen: helios bietet bereits ein einfaches Fenstermanagement, rudimentäre Eingabe-Verarbeitung, ein Logging-System sowie ein funktionierendes Rendering Layer samt Anbindung an die OpenGL-API.\\

