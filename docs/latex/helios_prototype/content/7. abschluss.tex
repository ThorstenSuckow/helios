\section{Abschluss und Ausblick}

Wir haben \textit{helios} vorgestellt, einen Prototypen für ein Game Framework zur Umsetzung eines \textit{Geometry Wars}-Klon.

Es folgte ein Überblick über die Projektdaten sowie über Aufbau und Architektur des C++-Projekts.
Dabei sind wir auf die Toolchain eingegangen und haben die verwendeten Bibliotheken von Drittanbietern vorgestellt.\\

Schwierigkeiten bei der Umsetzung haben wir in einer Diskussion reflektiert.
Aus diesen Reflexionen haben sich für uns interessante Fragen ergeben, die einer näheren Betrachtung wert sind:
Wie wirken sich verschiedene Formen der Game Loop~\cite[534 ff.]{Gre19} auf das \textit{Game Feel} aus?
Welchen Einfluss auf die Laufzeit haben die aus der Standardbibliothek übernommenen Datenstrukturen bei Neuallokation im \textit{Hot Path} des Renderings?
Und welchen Performancevorteil würden wir durch den Einsatz von \textit{raw pointern} im Gegensatz zu \textit{Smart Pointern} erzielen?\\

Aufgrund der konsequenten Anwendung allgemein anerkannter Entwicklungsprinzipien sehen wir derzeit wenig Bedarf an strukturellen Änderungen.
Es empfiehlt sich jedoch, hier keine voreiligen Schlüsse zu ziehen:
Bevor die technische Domäne vollständig durchdrungen ist, steht noch eine fachliche Auseinandersetzung mit den intrinsischen Eigenschaften von Kollisionsbehandlungsstrategien sowie (künstlicher) Gegnerintelligenz an.
Hier erwarten wir mehrere weitere Entwicklungsschleifen.
Im gleichen Zusammenhang betrachten wir auch den Entwurf der Rendering-Pipeline vorsichtig optimistisch.
Zwar folgt dieser klaren semantischen Strukturen und orientiert sich am \textit{mental model} eines Entwicklers.
Unsere Recherche hat jedoch gezeigt, dass datengetriebene Entwicklung~\cite{Bay22} und hardwarenahe Optimierungen durchaus zum Usus effizienter Spiele gehören.\\

Bei alldem soll nicht vergessen werden, dass wir in dieser frühen Phase des Projekts bewusst eine robuste Implementierung bevorzugt haben.
Wir blicken den nachfolgenden Iterationen daher gespannt entgegen.
Mit dem Projektziel ``16 ms pro Frame`` vor Augen schließen wir diesen Abschnitt mit einem Zitat von Donald E. Knuth:
\begin{quote}
    \centering
    \textit{``Premature optimization is the root of all evil.``}~\cite{Knu74}
\end{quote}
