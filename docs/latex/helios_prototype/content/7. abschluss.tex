\section{Abschluss und Ausblick}

Wir haben helios vorgestellt, einen Prototypen für ein Game Framework zur Umsetzung eines Geometry Wars Klon.

Es folgte ein Überblick über die Projektdaten und Aufbau und Architektur des C++-Projektes.
Hierbei sind wir auf die Toolchain eingegangen und haben verwendete Bibliotheken von Drittanbietern vorgestellt.\\

Schwierigkeiten bei der Umsetzung haben wir in einer Diskussion reflektiert.
Aus diesen Reflektionen haben sich für uns interessante Fragen ergeben, von denen wir denken, dass sie einer näheren Betrachtung wert sind: Wi wirken sich verschiedenen Formen der Game Loop~\cite[534 ff.]{Gre19} auf das Game Feel aus?
Welchen Einfluss auf die Laufzeit haben die aus der Standardbibliothek übernommenen Datenstrukturen bei Neuallokation im \textit{Hot Path} des Renderings.
Welchen Performancevorteil würden wir durch den Einsatz von \textit{Raw Pointern} im Gegensatz zu \textit{Smart Pointern} erhalten?\\

Aufgrund der konsequenten Anwendung von allgemein anerkannten Entwicklungsprinzipien sehen wir wenig Bedarf an strukturellen Änderungen.
Es empfiehlt sich jedoch, hier keine voreiligen Schlüsse zu ziehen: Bevor die technische Domäne komplett durchdrungen ist, steht noch eine fachliche Auseinandersetzung mit den intrinsischen Eigenschaften von Kollisionsbehandlungsstrategien sowie (künstlicher) Gegnerintelligenz an.
Hier sehen wir durchaus mehrere Entwicklungsschleifen.
In gleichem Zusammenhang betrachten wir deshalb auch den Entwurf der Rendering-Pipeline vorsichtig optimistisch.
Zwar folgt dieser klaren semantischen Strukturen und orientiert sich am \textit{Mental Model} eines Entwicklers.
Aber auch für diesen Teil hat die Recherche ergeben, dass effiziente Datenstrukturen und hardwarenahe Optimierungen durchaus zum Usus in der Spielentwicklung gehören - was gleichzeitig gute Kenntnisse über das verwendete Rendering-Backend voraussetzt.\\

Bei alldem soll nicht vergessen werden, dass wir in dieser Frühphase des Projektes bewusst eine robuste Implementierung bevorzugt haben.
Wir schauen deshalb den nachfolgenden Iterationen gespannt entgegen.
Mit dem Projektziel ``$16ms$  pro Frame`` vor Augen schließen wir deshalb zunächst - ruhigen Gewissens - mit dem Zitat von Donald E. Knuth: ``premature optimization is the root of all evil.``~\cite[]{Knu74}.

