\section{Introduction}

We are developing a \textit{Geometry Wars}\footnote{See~\cite[]{WikipediaGeometryWars}} clone in~C++.
In doing so, we partly base our implementation on a widely used game engine architecture described by Gregory in~\cite[]{Gre19}, while deliberately reducing the number of abstraction layers and omitting systems such as tooling, sound, and scripting (cf.~\cite[Figure~1.16, p.~39]{Gre19}).
The technical foundation therefore resembles a framework rather than a complete engine.
The actual game - conceived as a \textit{black box}~\cite[]{RB88} - is intended to use the framework's interfaces to access the required hardware and other subsystems.

\begin{figure}[tbp]
    \centering
    \includegraphics[width=0.5\columnwidth]{img/helios_logo}
    \caption{The \textit{helios} project logo. (Source: own representation)}
    \label{fig:helios_logo}
\end{figure}

The following sections present the architecture of the framework, referred to as \textbf{helios}\footnote{
    Helios, who “looks sharply about with shining eyes” (\textit{Homers Ilias}, Gesang~14, own translation), is the sun god in Greek mythology.
}, and discuss its prototypical development stage.
The software is being created within an agile \textit{Tracer Bullet Development} process~\cite[pp.~50~f.]{TH20}, which allows the entire system to be adapted at short notice if necessary.
From this, the architectural requirements are derived.
The project data and objectives are presented separately in Section~\ref{sec:projectdata}.

Furthermore, we discuss challenges and difficulties encountered during implementation.
These arise, among other factors, from the fact that the software - to be developed within a comparatively short time frame - is evaluated not only according to objective measures (such as the number of defects, software design choices, or object coupling), but also according to subjective criteria~\cite[385]{Bal08}.
Although there are no plans to classify the finished product according to usability quality characteristics such as ISO/IEC~9126\footnote{Usability quality according to ISO/IEC~9126,~\cite[466]{Bal08}; see also studies and investigations in~\cite[]{AZMK17} and~\cite[]{Ber10}.}, the result should not only be technically sound but also pleasant to use.
We summarize this user experience in general terms using the expression \textit{game feel}~\cite[]{Swi08}, established in contemporary game development.
