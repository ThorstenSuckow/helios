\section{Introduction}

We are developing a \textit{Geometry Wars}\footnote{See~\cite[]{WikipediaGeometryWars}} clone in C++.
Our implementation is partly based on the widely used game engine architecture described by Gregory~\cite[]{Gre19}, although we deliberately reduce the number of abstraction layers and omit systems such as tooling, sound, and scripting (cf.~\cite[Figure~1.16, 39]{Gre19}).
The resulting technical foundation therefore resembles a framework rather than a full-fledged engine.
The actual game - as a \textit{black box}~\cite[]{RB88} - is designed to use this framework’s interfaces to access hardware and other subsystems.

\begin{figure}[tbp]
    \centering
    \includegraphics[width=0.5\columnwidth]{img/helios_logo}
    \caption{The \textit{helios} project logo. (Source: own representation)}
    \label{fig:helios_logo}
\end{figure}

The following sections present the architecture of the framework referred to as \textbf{helios}\footnote{
    Helios, who ``looks around sharply with shining eyes`` (\textit{Homers Ilias}, Gesang 14, own translation), is the sun god in Greek mythology.
}.
We explicitly take into account its prototypical development status:
the software is being created as part of an agile \textit{Tracer Bullet Development} process~\cite[pp.~50-55]{TH20}, allowing the entire system to be adapted on short notice if necessary.
Requirements for the architecture are derived accordingly.
Project data and objectives are presented separately in Section~\ref{sec:projectdata}.

We also address the problems and challenges encountered during the implementation so far.
These arise, among other things, from the fact that the software product - developed within a comparatively short time frame - is to be evaluated not only on the basis of objective measures (such as the number of defects, software design decisions, or the degree of object coupling) but also on the basis of subjective criteria~\cite[385]{Bal08}.
Although there are no plans to classify the finished product according to usability quality characteristics such as those defined in ISO/IEC 9126\footnote{Usability quality according to ISO/IEC 9126 in~\cite[466]{Bal08}; see also studies and investigations in~\cite[]{AZMK17},~\cite[]{Ber10}.},
the result should not only be technically robust but also enjoyable to use.
In the following, we summarize this user experience in the more informal but established term \textit{game feel}~\cite[]{Swi08}, widely used in game development.
