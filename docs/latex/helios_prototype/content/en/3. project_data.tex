\section{Project Data and Goals}\label{sec:projectdata}

The objective of this project is to create a stable prototype of the twin-stick shooter\footnote{%
    For genre classification, see also~\cite[]{GameDeveloper}.%
} \textit{Geometry Wars} (see Figure~\ref{fig:geometry_wars}) using the programming language \textit{C++}23 and the OpenGL~API~\cite[]{OpenGLHomepage}.\\

\begin{figure}[tbp]
    \centering
    \includegraphics[width=\textwidth]{img/geometry_wars}
    \caption{\textit{Geometry Wars} first appeared in 2003 as an Easter egg in the game \textit{Project Gotham Racing} (Microsoft Game Studios).
    Due to its popularity, several sequels were released, the most recent being \textit{Geometry Wars~3: Dimensions Evolved} (Activision,~2016).
        (Source: Activision Publishing, Inc.)}
    \label{fig:geometry_wars}
\end{figure}

The following features were defined as project goals:

\begin{itemize}\itemsep0.5em
\item Playable 2D grid-based level
\item Time-attack mode with a duration of three minutes
\item High-score system
\item Controller-based input
\item Stable frame rate of at least 60 FPS with the simultaneous rendering of over a hundred enemies and objects
\item Three enemy types with simple AI (``spawn and chase'')
\item Enhanced game feel through graphical effects
\item Buildable and executable under Windows 11
\end{itemize}

While some criteria are quantitatively measurable (e.g., frame rate), others are intentionally left vague (e.g., game feel) to emphasize the exploratory nature of the project.
Formal requirements engineering is therefore not part of this work, particularly since the original game on which the implementation is based on also serves as a benchmark.

The learning process itself is considered a key objective of this work and is not evaluated by formal metrics.
Instead, the acquired experience and resulting insights are documented in this report and critically reflected upon.

Against this background, the chosen approach can be regarded as a form of dynamic requirements engineering ``with significantly greater degrees of freedom'' (own translation)~\cite[60]{MRP21}.
The aforementioned \textit{Tracer Bullet Development}, which primarily provides a platform for continuous integration, corresponds to the prototype described by~\textit{Pflug et al.} as a ``living lab``, which is iteratively evolved using agile methods (\textit{ibid.}, p.~61).\footnote{%
    Kasurinen et al.~\cite[]{KMS14} examine whether requirements management, rooted in traditional engineering, has a place in game development, which is strongly driven by creative processes.
    They show that many developers consciously derive parts of the requirements through user testing.
    Accordingly, games user research has become an important branch of the industry, helping studios understand and improve player experience (cf.~\cite[26]{Zam18}).
    Further discussions of software engineering in game development can be found in~\cite[]{KH09}.%
}

\subsection{Schedule}

The schedule and associated milestones are shown in Table~\ref{tab:schedule}.
The timeline serves as a guideline for implementation; the sequential order does not imply a waterfall process.

\setlength{\tabcolsep}{8pt}
\begin{table}[t]
    \centering
    {\renewcommand{\arraystretch}{1.2}%
        \begin{tabularx}{\textwidth}{@{} l l Y @{}}
            \toprule
            \textbf{Milestone} & \textbf{Date} & \textbf{Content} \\
            \midrule
            \texttt{milestone\_1} & 20 Oct 2025 &
            Provision of the application layer, including implementation of the event system,
            the input manager, and integration with the low-level API layer. \\
            \texttt{milestone\_2} & 17 Nov 2025 &
            Provision of the rendering engine; initial design of the playing field, including display of the player's ship. \\
            \texttt{milestone\_3} & 22 Dec 2025 &
            Implementation of physics and player input for ship control; firing mechanics. \\
            \texttt{milestone\_4} & 19 Jan 2026 &
            Implementation of the core game rules and mechanics; playable prototype. \\
            \texttt{milestone\_5} & 09 Feb 2026 &
            Prototype refinement and fine-tuning. \\
            \texttt{milestone\_6} & 16 Mar 2026 &
            Submission of documentation and project presentation. \\
            \bottomrule
        \end{tabularx}}
    \caption{Planned milestones for the implementation of the \textit{Geometry Wars} clone.
    At the time of publication of this document, the first milestone has been completed:
    \textit{helios} already provides simple window management, rudimentary input processing,
        a logging system, and a functional rendering layer interfacing with the OpenGL API.}
    \label{tab:schedule}
\end{table}

\noindent
Since the gameplay concept to be implemented is already specified, no preliminary prototyping is required.
The current development phase is classified as the \textit{hard-architecture design} phase, following Rollings and Morris~\cite[628]{RM04}.

The project is published on GitHub under the MIT license; both the source code and the issue tracker are publicly accessible~\cite[]{heliosgithub}.
