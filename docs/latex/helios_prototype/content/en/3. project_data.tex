\section{Project Data and Goals}\label{sec:projectdata}

The goal of this project is to create a stable prototype of the twin-stick shooter\footnote{
    For genre classification, see also~\cite[]{GameDeveloper}.
} \textit{Geometry Wars} (see Figure~\ref{fig:geometry_wars}) using the C++23 programming language and the OpenGL API~\cite[]{OpenGLHomepage}.\\

\begin{figure}[tbp]
    \centering
    \includegraphics[width=1\columnwidth]{img/geometry_wars}
    \caption{\textit{Geometry Wars} first appeared in 2003 as an Easter egg in the game \textit{Project Gotham Racing 2} (Microsoft Game Studios). Due to its great popularity, several sequels were released, most recently in 2016 with \textit{Geometry Wars 3: Dimensions Evolved} (Activision). (Source: Activision Publishing, Inc.)}
    \label{fig:geometry_wars}
\end{figure}

The following features were agreed upon for implementation:

\begin{itemize}
    \itemsep0.5em
    \item Playable level (2D grid)
    \item Time attack mode with a duration of 3 minutes
    \item High score system
    \item Controller support
    \item Stable frame rate of at least 60 FPS with simultaneous display of over one hundred enemies and objects
    \item Three enemy types with simple AI (``spawn and chase``)
    \item Enhancement of the game feel through graphical effects
    \item Compilable and executable under Windows~11
\end{itemize}

While some criteria are specifically measurable (e.g., frame rate), other requirements are deliberately vague (e.g., game feel) and underscore the exploratory nature of the project.
Formal requirements engineering is therefore not part of this work, especially since the original game on which the implementation is based also serves as a reference and benchmark.

The learning process itself is considered a key objective of this project; no formal metrics are used for its evaluation.
Instead, the experiences and results obtained are documented in writing, and the present report will serve as a critical reflection of them at the end of the project.


Against this background, the approach can be regarded as a form of dynamic requirements engineering ``with significantly greater degrees of freedom``~\cite[60]{MRP21} (own translation).
The aforementioned \textit{Tracer Bullet Development} process, which primarily provides a platform for integration, corresponds to the prototype referred to by Pflug et~al. as a ``living lab``, which is further developed using agile methods (\textit{ibid., p.~61}), allowing for the contribution of original creative ideas.\footnote{
    In~\cite[]{KMS14}, Kasurinen et~al.\ explore whether requirements management, which is rooted in engineering, has a place in game development, a field strongly influenced by creative processes. They show that some requirements are consciously derived through \textit{user testing}. Accordingly, games user research has become an important branch of the games industry that helps developers understand and improve the player experience (\cite[p.~26]{Zam18}). Further discussions regarding software engineering in game development can be found in Kanode and Haddad~\cite[]{KH09}.
}

\subsection{Schedule}

The schedule and associated milestones are presented in Table~\ref{tab:schedule}.
The timeline serves as a guideline for implementation; the sequential listing does not imply a waterfall model.

\setlength{\tabcolsep}{8pt}
\begin{table}[t]
    \centering
    {\renewcommand{\arraystretch}{1.2}%
        \begin{tabularx}{\textwidth}{@{} l l Y @{}}
            \toprule
            \textbf{Milestone} & \textbf{Date} & \textbf{Content} \\
            \midrule
            \texttt{milestone\_1} & 2025-10-20 &
            Provision of the application layer, including implementation of the event system,
            the input manager, and connection to the low-level API subsystem. \\
            \texttt{milestone\_2} & 2025-11-17 &
            Provision of the rendering engine; initial design of the playing field, including display of the player's ship. \\
            \texttt{milestone\_3} & 2025-12-22 &
            Implementation of physics and player input for ship control; fire mechanics. \\
            \texttt{milestone\_4} & 2026-01-19 &
            Implementation of the core game rules and mechanics; playable prototype. \\
            \texttt{milestone\_5} & 2026-02-09 &
            Provision of the prototype; fine-tuning. \\
            \texttt{milestone\_6} & 2026-03-16 &
            Submission of documentation and presentation of the project. \\
            \bottomrule
        \end{tabularx}}
    \caption{Planned milestones for the implementation of the \textit{Geometry Wars} clone.
    At the time of publication of this document, the first milestone has been completed: \textit{helios} already provides simple window management, rudimentary input processing, a logging system, and a functional rendering layer with a connection to the OpenGL API.}
    \label{tab:schedule}
\end{table}

\noindent
Since the game concept to be implemented is already defined, there is no need for an exploratory prototype phase.
The current stage of development can be classified as the \textit{hard-architecture design} phase, following Rollings and Morris~\cite[pp.~628-630]{RM04}.

The project is published under the MIT license on GitHub, with both the source code and the issue tracker publicly accessible~\cite[]{heliosgithub}.
