\section{Project Structure}

We present the structure of the project and the associated toolchain.
We also briefly introduce the contents of the individual directories.\\

The top level of the \textit{helios} directory structure provides access to tests, executable sample programs, benchmarks, documentation, and source files.
The chosen structure follows the conventions for C++ projects defined by the \textit{Pitchfork} project~\cite[]{Pitchfork}.
The directory names are self-explanatory and are shown in Figure~\ref{fig:directory_structure}.

\begin{figure}[htbp]
    \setlength{\DTbaselineskip}{18pt}
    \dirtree{%
        .1 ./.
        .2 benchmarks/.
        .2 docs/.
        .2 examples/.
        .2 include/.
        .2 src/.
        .2 tests/.
    }
    \caption{Project directory of \textit{helios} (excerpt).}
    \label{fig:directory_structure}
\end{figure}

The project comprises several build stages to compile the source files and generate the tests and example programs.
For automation, we use \texttt{CMake}~\cite[]{CMake}, which also allows us to perform simple dependency management.
The third-party libraries (TPLs) required for development (including \texttt{glfw}~\cite[]{glfwHomepage} and \texttt{glad}~\cite[]{gladgithub}) can thus be fetched directly from external sources, statically compiled, and integrated (see Listing~\ref{lst:cmake}).
This approach enables us to automate the project’s preconfiguration and avoid manual (and often error-prone~\cite[]{FG22}) integration of TPLs.

\vspace{4mm}
\begin{lstlisting}[style=c++style,
    caption={Excerpt from helios' CMakeLists.txt.
    This section declares and fetches GLFW v3.4 and GLAD v2.0.8 via \texttt{FetchContent} from the respective GitHub repositories (URLs omitted for clarity).
    Subsequently, a GLAD loader for OpenGL 4.6 is created as a static library.},
    label=lst:cmake]
...
FetchContent_Declare(glfw
    GIT_REPOSITORY [url]
    GIT_TAG        3.4
)

FetchContent_Declare(glad
    GIT_REPOSITORY [url]
    GIT_TAG        v2.0.8
    SOURCE_SUBDIR cmake
)

FetchContent_MakeAvailable(glfw glad)

# GLAD v2: Core GL 4.6
glad_add_library(
    glad_gl_core_46 STATIC REPRODUCIBLE LOADER API 
    gl:core=4.6
)
...
\end{lstlisting}

\subsection{Directory Contents}

In the following, we present the main directories in alphabetical order and briefly discuss their contents.

\subsection*{\texttt{/benchmarks}}

For benchmarking individual functions, \textit{helios} uses \textit{Google Benchmark}~\cite[]{googlebenchmarkgithub}.
Currently, there are benchmarks for mathematical types and functions.
We use these to compare our implementations with those of \texttt{glm}~\cite[]{glmGithub}, which serves as a reference.
This allows us to identify potential bottlenecks in the \textit{Game Loop} at an early stage, for example in operations involving affine transformation matrices.
Further benchmarks - measuring the performance of selected functions in the \textit{Rendering Pipeline} or \textit{Application Stage}~\cite[687]{Gre19} (e.g., during \textit{Culling}) - are planned.

\setlength{\tabcolsep}{8pt}
\begin{table}[t]
    \centering
    {\renewcommand{\arraystretch}{1.2}%
        \begin{tabular}{lrrr}
            \hline
            \textbf{Benchmark} & \textbf{Time} & \textbf{CPU} & \textbf{Iterations} \\
            \hline
            BM\_mat4Constructor/real\_time & 5.81 ns & 5.85 ns & 106,874,145 \\
            BM\_mat4Multiply/real\_time    & 175 ns  & 168 ns  & 3,896,895 \\
            \hline\\
        \end{tabular}}
    \caption{Sample output of benchmark results based on our custom \texttt{mat4} implementation.}
    \label{tab:mat4-benchmark}
\end{table}

No evaluation of alternative benchmarking frameworks was conducted; the choice was based on popularity and community recommendations.

\subsection*{\texttt{/docs}}

In addition to formal documentation\footnote{Such as the \LaTeX\ source code of this document.},
this directory contains API documentation generated from the source files using \texttt{doxygen}~\cite[]{Doxygen}.
A key advantage of \texttt{doxygen} is its ability to export to multiple formats.
This allows us to generate HTML or XML documentation and integrate it into the \textit{helios} project website~\cite[]{helios} via a dedicated build step.

A formal evaluation of documentation tools was not performed.
The selection was guided by ease of integration and widespread adoption.
We also consider the extensive range of output formats (including PDF and Markdown) beneficial for maintaining project documentation.

\subsection*{\texttt{/tests}}

For unit testing, \textit{helios} employs \textit{Google Test}~\cite[]{googletestgithub}.
To increase development speed, especially during the early and potentially unstable project phase, we initially decided against pursuing high test coverage.
Current tests therefore focus on mathematical functions - for instance, validating transformations within the scene graph.
\textit{helios} uses \texttt{glm} as a test oracle~\cite[pp.~917-919]{Bin99} for comparison purposes.\footnote{
    Alternatively, \texttt{glm} can be regarded as a \textit{baseline} against which \textit{helios} functions (as a \textit{delta version}) are compared in regression tests.
    We acknowledge that any errors in \texttt{glm} would propagate to \textit{helios}; however, given the maturity of the \textit{glm} project, this risk is considered minimal.}

As with benchmarking, no formal evaluation of unit testing frameworks was conducted.
The decision was based on popularity and recommendations.

\subsection*{\texttt{/examples}}

This directory contains example programs that demonstrate the individual functionalities of the framework.
During development, these programs help define functionalities in a \textit{top-down approach} and refine them iteratively~\cite[]{Wir71}.
This approach allows us to focus on the required interfaces of individual subsystems without getting lost in implementation details that can be added in later iterations.

\begin{figure}[!h]
    \centering
    \includegraphics[width=1\columnwidth]{img/cube_example}
    \caption{Screenshot of the \texttt{simple\_cube\_rendering} demo, used in a top-down approach to incrementally implement the functionality defined in the first milestone. (Source: own recording)}
    \label{fig:simple-cube-rendering-demo}
\end{figure}

\subsection*{\texttt{/include, /src}}

With C++23, \textit{helios} implements its classes and functions entirely through Module Interface and Implementation Units~\cite[pp.~211-213]{Str24}.
These are stored in \texttt{/include} and \texttt{/src}, respectively.
Modules not divided into Interface and Implementation Units reside exclusively in \texttt{/include} as ``header-only`` components.
These include classes such as \texttt{helios::math::vec3}, which represents a three-dimensional vector type whose functions are predominantly declared as \texttt{constexpr}\footnote{
    The \texttt{constexpr} keyword allows expressions to be evaluated at compile time, requiring their complete definition at translation time~\cite[p.~330]{Gre24}.}.
The directories are further subdivided into \texttt{ext} and \texttt{helios}:
\begin{itemize}
    \itemsep0.5em
    \item \texttt{ext} contains platform-specific implementations of interfaces contractually defined by the \textit{helios} framework (hardware-related window and application abstractions, as well as rendering pipeline specifics).
    \item \texttt{helios} contains the actual framework code.
\end{itemize}
