\section{Conclusion and Outlook}

We have presented \textit{helios}, a prototype game framework developed in C++ for implementing a \textit{Geometry Wars} clone.
We discussed the project data and toolchain, introduced the third-party libraries used, and provided an overview of the project’s structure and architecture.

In the discussion section, we reflected on several challenges encountered during implementation.
These reflections raise questions that merit further investigation:
How do different forms of the game loop~\cite[pp.~534-538]{Gre19} affect the game feel?
What influence do data structures from the standard library have on runtime when reallocated within the \textit{hot path} of rendering?
And what measurable performance gain might result from using \textit{raw pointers} instead of \textit{smart pointers}?

Thanks to the consistent application of well-established development principles, we currently see little need for structural changes.
Nevertheless, it would be premature to draw definitive conclusions:
Before the technical domain is fully understood, a deeper examination of the intrinsic properties of collision-handling strategies and (artificial) opponent intelligence is still required.
We therefore expect several further development iterations in these areas.

In the same context, we remain cautiously optimistic about the design of the rendering pipeline.
Its layout follows clear semantic structures and is based on a conceptual idea (\textit{mental model}) of data processing.
However, optimal preparation and transfer of rendering instructions to the rendering backend are equally crucial for achieving high performance.
Here, our design will have to prove its efficiency in practical application.

Furthermore, our research indicates that DOD\footnote{\textit{Data-Oriented Design}}-based systems commonly form the foundation of efficient game engines~\cite{Bay22}.
These systems do not rely solely on classical OOP paradigms but employ data-oriented approaches in performance-critical subsystems, where flat component structures according to the \textit{ECS\footnote{\textit{Entity-Component System}} pattern}~\cite[]{RCCK25} can provide significant performance benefits~\cite[]{WWM22}.\footnote{
    See also the data-oriented framework \textit{ECS for Unity}~\cite{UnityECS} in this context.
    Unreal Engine likewise offers \textit{MassEntity}~\cite{UnrealMassEntity}, a ``gameplay-focused framework for data-oriented calculations``.
}
Although we are not currently pursuing this approach, it remains valuable to consider it as an established paradigm in high-performance game engine design.
Future optimizations could incorporate concepts inspired by this methodology.

With all this in mind, it should not be forgotten that we have deliberately prioritized robustness over optimization at this early stage of development.
We therefore look forward to subsequent iterations with anticipation.
Keeping the project goal of ``16 ms per frame`` in view, we conclude this document with a quote by Donald E. Knuth:
\begin{quote}
    \centering
    \textit{``Premature optimization is the root of all evil.``}~\cite{Knu74}
\end{quote}
