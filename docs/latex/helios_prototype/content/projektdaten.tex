\section{Projektziele und -daten}\label{sec:projektdaten}

Das Ziel der Projektarbeit ist es, einen stabil laufenden Prototyp des Twin-Stick Shooters\footnote{
Zur Genreeinteilung siehe auch \cite[]{GameDeveloper}
} \textit{Geometry Wars} (siehe Abbildung~\ref{fig:geometry_wars}) auf Basis der Programmiersprache \textit{C++} (23) und der OpenGL-API\footnote{
    \url{https://www.khronos.org/opengl} (abgerufen 23.10.2025)
}  zu erstellen.\\

\noindent
Als grundsätzlich fertigzustellende Kernfeatures wurden vereinbart (siehe Anhang~\ref{sec:projektvorschlag}):

\vspace{2mm}
\begin{itemize}
    \itemsep0.5em
    \item Spielbares Level (2D-Grid)
    \item Time-Attack Modus mit einer Dauer von 3 Minuten
    \item Highscore-System
    \item Controller-Steuerung
    \item Stabile Framerate von mindestens 60 FPS bei gleichzeitiger Darstellung von über hundert Gegnern und Objekten
    \item Drei Gegnertypen mit einfacher KI (``Spawn and Chase``)
    \item Aufwertung des Game Feels durch grafische Effekte
    \item Compilierbar und lauffähig unter Windows 11
\end{itemize}
\vspace{2mm}

Aus den aufgeführten Anforderungen gehen nur äußerst unscharf formulierte Anforderungen hervor, aus denen sich keine konkreten Abnahmekriterien schließen lassen.
Ein \textit{Requirements Engineering} ist folglich auch kein Teil dieser Projektarbeit: Als ein Maß dient neben den o.a. funktionalen Kriterien vor allem auch das Original-Spiel, an dem sich die Umsetzung orientiert.
Insbesondere aber legen wir für die Projektarbeit als ein wesentliches Ziel den \textit{Lernprozess} fest, der sich zwar nicht direkt messen lässt, dessen Ergebnisse aber zumindest in einem abschließenden Dokument festgehalten werden können.

In diesem Fall sehen wir uns einem dynamischen Requirements Engineering ``mit deutlich größeren Freiheitsgeraden``~\cite[60]{MRP21} gegenüber: Das von uns angesprochene \textit{Tracer Bullet Development}, das uns vor allem eine Plattform zur Integration bereitstellt, entspricht dem von \textit{Pflug et al.} mit ``Living Lab`` bezeichneten Prototyp, der also nach agilen Methoden entwickelt wird (\textit{ebd., S. 61})\footnote{
    \textit{Kasurinen et al.} gehen in~\cite[]{KMS14} der Frage nach, ob das im Ingenieursbereich angesiedelte Anforderungsmanagement bei der stark durch kreative Prozesse beeinflussten Spieleentwicklung einen Platz hat. Sie zeigen, dass befragte Entwickler einen Teil der Anforderungen an ihr Spiel ganz bewusst erst durch \textit{User Testing} ableiten. Entsprechend ist \textit{Games User Research} ein wichtiger Zweig in der Spieleindustrie, der den Unternehmen dabei hilft, Spiele-Erfahrung zu verstehen und zu verbessern (vgl.~\cite[26]{Zam18})
}.\\

\subsection{Zeitplan}

Wir stellen den Zeitplan vor und die aus dem Projektvorschlag übernommenen Meilensteinen, die als Orientierung zur Umsetzung der benötigten Features dienen\footnote{Aus der sequentiellen Auflistung soll kein Rückschluss auf ein Wasserfall-Modell gezogen werden}:

\vspace{2mm}
\begin{itemize}
    \itemsep0.5em
    \item \textbf{milestone\_1}: 20.10.2025: Bereitstellung der Applikationsschicht inkl. Implementierung des Event
    Systems, Input Managers sowie Anbindung an das Low Level API-Sub-Systems
    \item \textbf{milestone\_2}: 17.11.2025: Bereitstellung der Rendering Engine. Erster Entwurf des Spielfeldes
    samt Darstellung des Spielerschiffs.
    \item \textbf{milestone\_3}: 22.12.2025: Umsetzung der Physik und Spielereingabe zur Steuerung des Schiffs,
    Feuermechaniken
    \item \textbf{milestone\_4}: 19.01.2026: Implementierung der wichtigsten Spielregeln- und Mechaniken. Spiel
    fähiger Prototyp.
    \item \textbf{milestone\_5}: 09.02.2026: Bereitstellung des umgesetzten Prototyps. Feinschliff.
    \item \textbf{milestone\_6}: 16.03.2026: Abgabe der Dokumentation und Vorstellung des Projekts.
\end{itemize}
\vspace{2mm}

\noindent
Mit der Vorgabe des umzusetzenden Spielkonzeptes entfällt für uns dessen Prototyping.
Wir folgen \textit{Rollings und Morris} und schließen, dass wir uns zum Zeitpunkt der Umsetzung des ersten Meilensteins in der \textit{hard-architecture design}-Phase befinden (vgl.~\cite[628]{RM04}), der zum Zeitpunkt der Veröffentlichung dieses Dokumentes abgeschlossen ist: helios bietet bereits ein einfaches Fenstermanagement, rudimentäre Eingabe-Verarbeitung, ein Logging-System sowie ein funktionierendes Rendering Layer samt Anbindung an die OpenGL-API.\\

Das Projekt ist \textit{MIT} lizenziert auf Github veröffentlicht, Source Code und Ticketsystem sind öffentlich einsehbar~\cite[]{heliosgithub}.\\
